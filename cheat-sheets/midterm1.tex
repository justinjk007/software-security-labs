%%%%%%%%%%%%%%%%%%%%%%%%%%%%%%%%%%%%%%%%%%%%%%%%%%%%%%%%%%%%%%%%%%%%%%
% writeLaTeX Example: A quick guide to LaTeX
%
% Source: Dave Richeson (divisbyzero.com), Dickinson College
%
% A one-size-fits-all LaTeX cheat sheet. Kept to two pages, so it
% can be printed (double-sided) on one piece of paper
%
% Feel free to distribute this example, but please keep the referral
% to divisbyzero.com
%%%%%%%%%%%%%%%%%%%%%%%%%%%%%%%%%%%%%%%%%%%%%%%%%%%%%%%%%%%%%%%%%%%%%%

\documentclass[10pt,landscape]{article}
\usepackage{amssymb,amsmath,amsthm,amsfonts}
\usepackage{multicol,multirow}
\usepackage{calc}
\usepackage{ifthen}
\usepackage[landscape]{geometry}
\usepackage[colorlinks=true,citecolor=blue,linkcolor=blue]{hyperref}
\usepackage{enumitem}
\setlist[itemize]{leftmargin=*}
\setlist{noitemsep}
\usepackage{listings}
\lstset{
backgroundcolor=\color{white},   % choose the background color; you must add \usepackage{color} or \usepackage{xcolor}; should come as last argument
basicstyle=\scriptsize,          % the size of the fonts that are used for the code
breakatwhitespace=false,         % sets if automatic breaks should only happen at whitespace
breaklines=true,                 % sets automatic line breaking
captionpos=b,                    % sets the caption-position to bottom
commentstyle=\bfseries,          % comment style
extendedchars=true,              % lets you use non-ASCII characters; for 8-bits encodings only, does not work with UTF-8
firstnumber=1000,                % start line enumeration with line 1000
frame=single,                    % adds a frame around the code
keepspaces=true,                 % keeps spaces in text, useful for keeping indentation of code (possibly needs columns=flexible)
keywordstyle=\color{blue},       % keyword style
language=bash,                   % the language of the code
morekeywords={ls,ping},          % if you want to add more keywords to the set
numbers=left,                    % where to put the line-numbers; possible values are (none, left, right)
numbersep=5pt,                   % how far the line-numbers are from the code
numberstyle=\tiny\color{mygray}, % the style that is used for the line-numbers
rulecolor=\color{black},         % if not set, the frame-color may be changed on line-breaks within not-black text (e.g. comments (green here))
showspaces=false,                % show spaces everywhere adding particular underscores; it overrides 'showstringspaces'
showstringspaces=false,          % underline spaces within strings only
showtabs=false,                  % show tabs within strings adding particular underscores
stepnumber=2,                    % the step between two line-numbers. If it's 1, each line will be numbered
stringstyle=\color{black},       % string literal style
tabsize=2,                       % sets default tabsize to 2 spaces
title=\lstname                   % show the filename of files included with \lstinputlisting; also try caption instead of title
}

\ifthenelse{\lengthtest { \paperwidth = 11in}}
    { \geometry{top=.5in,left=.5in,right=.5in,bottom=.5in} }
    {\ifthenelse{ \lengthtest{ \paperwidth = 297mm}}
    {\geometry{top=1cm,left=1cm,right=1cm,bottom=1cm} }
    {\geometry{top=1cm,left=1cm,right=1cm,bottom=1cm} }
    }
\pagestyle{empty}
\makeatletter
\renewcommand{\section}{\@startsection{section}{1}{0mm}%
    {-1ex plus -.5ex minus -.2ex}%
    {0.5ex plus .2ex}%x
    {\normalfont\large\bfseries}}
\renewcommand{\subsection}{\@startsection{subsection}{2}{0mm}%
    {-1explus -.5ex minus -.2ex}%
    {0.5ex plus .2ex}%
    {\normalfont\normalsize\bfseries}}
\renewcommand{\subsubsection}{\@startsection{subsubsection}{3}{0mm}%
    {-1ex plus -.5ex minus -.2ex}%
    {1ex plus .2ex}%
    {\normalfont\small\bfseries}}
\makeatother
\setcounter{secnumdepth}{0}
\setlength{\parindent}{0pt}
\setlength{\parskip}{0pt plus 0.5ex}
% -----------------------------------------------------------------------

\title{Midterm 1 cheat sheet}

\begin{document}

\raggedright
\footnotesize

\begin{center}
    \Large{\textbf{Midterm 1 cheat sheet}} \\
\end{center}
\begin{multicols}{3}
\setlength{\premulticols}{1pt}
\setlength{\postmulticols}{1pt}
\setlength{\multicolsep}{1pt}
\setlength{\columnsep}{2pt}

\section{Chapter 01}
\textbf{NIST: }National Institute of Standards and Technology

\section{Chapter 02}
Many symmetric block encryption algorithms, including DES, have a structure first described by
\textbf{Horst Feistel} of IBM in 1973
\subsection{Data Encryption Standard (DES)}
The plaintext is 64 bits and the key is 56 bits
\begin{itemize}
\item There are 16 rounds of processing.
\item From the original 56-bit key, 16 subkeys are generated
\end{itemize}
\subsection{Triple DES (Can use 2 keys or 3 keys)}
The function follows an encrypt-decrypt-encrypt (EDE) sequence: C = E(K3, D(K2, E(K1, P)))
\subsection{Advanced Encryption Standard (AES)}
NIST called for proposals for a new AES in 1997

Block size is 128, key size can be 128, 192, 256
\begin{itemize}
\item Selected Rijndaelin November 2001
\item Published as FIPS 197
\item 10 rounds: the first 9 has 4 stages each and the 10th round has only 3 stages
\item Each round has one of permutation and three of substitution
    stages: \textbf{1}.Substitute Bytes: Uses a table (called S-box), to
    perform a byte-by-byte substitution of the block \textbf{2}.Shift
    Rows (On encryption left rotate each row of State by 0,1,2,3 bytes
    respectively) \textbf{3}.Mix Columns \textbf{4}.Add Round key: A
    simple bitwise XOR of the current block with a portion of the
    expanded key
\end{itemize}
\textbf{Electronic Codebook} (ECB): Plaintext is handled b bits at a
time and each block is encrypted using the same key

\subsection{RC4(Stream Cipher)(Rivest Cipher4 )}
Normally 64 or 128 but keys, text cipher 1 byte at a time
KSA: Key scheduling algorithm
\lstset{language=bash,label= ,caption= ,captionpos=b,numbers=none}
\begin{lstlisting}
for
    i = 0 to 255 do S[i] = i;
T[i] = K[i mod k - len];
\end{lstlisting}

\lstset{language=bash,label= ,caption= ,captionpos=b,numbers=none}
\begin{lstlisting}
j = 0;
for
    i = 0 to 255 do
    {
    j = (j + S[i] + T[i])mod 256;
    Swap(S[i], S[j]);
    }
\end{lstlisting}

Pseudo Random Generation algorithm
\lstset{language=bash,label= ,caption= ,captionpos=b,numbers=none}
\begin{lstlisting}
i, j = 0;
while (true)
    i = (i + 1)mod 256;
j = (j + S[i])mod 256;
Swap(S[i], S[j]);
t = (S[i] + S[j])mod 256;
k = S[t];
\end{lstlisting}
Finally, encrypt using \textbf{XOR}

\begin{center}
\begin{tabular}{ |c|c|c| } 
 \hline
 A & B & A x B \\ 
 \hline
 0 & 0 & 0 \\ 
 0 & 1 & 1 \\ 
 1 & 0 & 1 \\ 
 1 & 1 & 0 \\ 
 \hline
\end{tabular}
\end{center}

\begin{itemize}
\item To encrypt XOR the value k with the next byte of the plain text
\item To decrypt, XOR the value k with the next byte of ciphertext
\end{itemize}

\textbf{Cipher Block Chaining} The input to the encryption algorithm is the XOR of 64 bits of
plaintext and the previous 64 bits of ciphertext. At start uses initialaztion vector instead or cipher text.
\lstset{language=bash,label= ,caption= ,captionpos=b,numbers=none}
\[C_j= E(K, [C_j–1 \oplus P_j])\]

\textbf{Counter mode(CTR)} The counter is initialized to some value and then incremented by 1 for
each subsequent block (modulo 2b, where b is the block size). For encryption, the counter is
encrypted and then XORed with the plaintext block

~\\
\textbf{Block reordering is a threat to symmetric encryption}

~\\

\textbf{MAC:} Message authentication code. It uses a secret key to
generate a small block of data, known as a MAC, that is appended to
the message. The recipient performs the same calculation on the
received message, using the same secret key, to generate a new message
authentication code. If the MAC generated on the other side matches,
message was not altered.

DES can be used to generate an encrypted version of the message, and
the last number of bits of ciphertext are used as the code.

\textbf{One way hash functions:} Hash functions are used in other Internet protocols such as Transport
Layer Security \textbf{(TLS)} and Secure Electronic Transaction \textbf{(SET)}

\textbf{HMAC:} Keyed-Hash message authentication code.

\subsection{Public Key cryptography}
\textbf{RSA} By Rivest, Shamir \& Adlemanof MIT in 1977

\lstset{language=bash,label= ,caption= ,captionpos=b,numbers=none}
\begin{lstlisting}
# Choose two distinct prime numbers, such as
p = 61 and q = 53
# Compute n = pq giving
n = 61 x 53 = 3233
# Compute the Carmichael's totient (lamda)
lamda(n) = lcm(p-1, q-1) giving
lamda(3233) = lcm(60,52) = 780
# Choose any number 1 < e < 780 that is coprime to 780.
# only number that divides both of them should be 1
Let e = 17
# Compute d
d x e = 1 mod lamda (n)
# d x 17 mod 780 should get the reminder 1
d = 413
# The public key is (n = 3233, e = 17)
c(m) = m^17 mod 3233
# The private key is (n = 3233, d = 413)
m(c) = c^413 mod 3233
\end{lstlisting}

\textbf{Public-key certificate} It consists of a public key plus a
user ID of the key owner, with the whole block signed by a trusted
third party.
One scheme has become universally accepted for formatting public-key certificates:\textbf{the X.509 standard.} 

\textbf{Diffie-Hellman Key Exchange}


\subsection{Page 2 section}
Lorem ipsum dolor sit amet, consectetur adipiscing elit. Donec blandit
tristique hendrerit. In tempus mi mauris, sed hendrerit lorem lobortis
non. Aliquam erat volutpat. Etiam ornare aliquet suscipit. Proin
euismod pellentesque facilisis. Proin euismod sapien ligula, quis
congue velit placerat ut. Fusce ex erat, aliquet id nibh eu, euismod
porttitor magna. In orci mi, tempor nec rhoncus eu, congue non
ligula. Aenean non sodales risus, iaculis viverra felis. Nunc
condimentum commodo mauris non ullamcorper.

Quisque hendrerit, mi eu semper viverra, quam leo fermentum nisi, vel
hendrerit ante risus eget tellus. Maecenas imperdiet bibendum tellus,
ut tempus augue faucibus quis. Fusce cursus pretium lobortis. Quisque
fringilla dignissim nibh, ut interdum mi viverra sit amet. Proin
feugiat erat odio, id consectetur nisi iaculis sed. Suspendisse
potenti. Integer vel mauris vitae mi aliquam convallis ac ullamcorper
quam. Praesent elementum egestas nulla, mollis cursus sem venenatis
eu. Morbi pulvinar sed neque pharetra varius.

Duis in lectus est. Nunc sollicitudin neque eget viverra
egestas. Mauris ornare ligula magna, vitae posuere risus pulvinar
quis. Sed at leo ut magna lobortis tincidunt. Duis vitae mauris vitae
ligula fringilla suscipit. Morbi non orci a lacus vehicula varius
tristique in lorem. Phasellus nec eros nec eros varius imperdiet quis
id enim. Pellentesque tempor dui non odio egestas blandit. Morbi a
massa pretium, feugiat arcu a, malesuada felis. In vitae venenatis
neque. Duis posuere ipsum at nibh finibus venenatis eu ac mi. Nullam
pulvinar turpis sem, in ultrices sapien accumsan id. Aenean non mi ac
ex semper aliquam. Nam at commodo nisl. Aliquam ac metus est. Sed eget
enim convallis, hendrerit nibh quis, suscipit risus.

Quisque dignissim turpis quis nisl pellentesque, a molestie nulla
efficitur. Aliquam erat volutpat. In hac habitasse platea
dictumst. Maecenas maximus erat mollis vulputate iaculis. Orci varius
natoque penatibus et magnis dis parturient montes, nascetur ridiculus
mus. Sed dui ligula, tincidunt ac tempus et, viverra posuere
magna. Phasellus luctus purus in laoreet ultricies. Donec vehicula
eget purus facilisis placerat. Fusce vel purus et dui consectetur
lacinia. Nullam sollicitudin commodo nibh non sollicitudin. Mauris
rutrum vitae ante eget hendrerit. Duis eleifend faucibus diam in
mattis.

~\\

\vfill
\hrule
~\\
\textbf{Justin Kaipada, 100590167}
\end{multicols}

\end{document}